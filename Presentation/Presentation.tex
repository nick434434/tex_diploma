\documentclass[10pt]{beamer}
\usepackage[russian]{babel}
\usepackage[utf8]{inputenc}
\usepackage[T2A]{fontenc}
\inputencoding{utf8}
\usepackage{amssymb,amsfonts,amsmath,mathtext,amsthm,mathtools}
\usepackage{float}
\usepackage[dvips]{graphicx}
\usepackage[unicode, pdfborder = {0 0 0}, colorlinks, linkcolor = black]{hyperref}
\usepackage{sansmathaccent}
\pdfmapfile{+sansmathaccent.map}
\usetheme{Frankfurt}
% \usecolortheme{seahorse}
\title{Восстановление циклически смазанных изображений с вырожденным смазом}
\author[Ковалев~Н.~Е.]{Ковалев~Никита~Евгеньевич\\научный руководитель -\\доц., к.ф.-м.н., Козак~Анатолий~Всеволодович}
\institute[ЮФУ, ИММиКН]{Южный федеральный университет\\ Институт математики, механики и компьютерных наук им. И.И. Воровича}
\date{Ростов-на-Дону, 2018}


\DeclarePairedDelimiter\ceil{\lceil}{\rceil}
\DeclarePairedDelimiter\floor{\lfloor}{\rfloor}

\setcounter{tocdepth}{5}
\setcounter{secnumdepth}{5}

\newcounter{qcounter}


\makeatletter
\defbeamertemplate*{footline}{my theme}{
	\leavevmode%
	\hbox{%
	\begin{beamercolorbox}[wd=.3\paperwidth,ht=2.25ex,dp=1ex,center]{author in head/foot}%
		\usebeamerfont{author in head/foot}%
		\insertshortauthor~~\beamer@ifempty{\insertshortinstitute}{}{(\insertshortinstitute)}
	\end{beamercolorbox}%
	\begin{beamercolorbox}[wd=.7\paperwidth,ht=2.25ex,dp=1ex,center]{title in head/foot}%
		\usebeamerfont{title in head/foot}\inserttitle
	\end{beamercolorbox}}%
}
\makeatother


\begin{document}

\begin{frame}
\maketitle
\end{frame}

\begin{frame}
\LARGE{Содержание}
\large{\tableofcontents}
\end{frame}


\section{Цели работы}
\begin{frame}
\begin{block}{\LARGE{Цели работы}}
\begin{list}{\arabic{qcounter})~}{\usecounter{qcounter}}
\item Исследовать матрицу смаза, ее свойства
\item Изучить возможность упрощения задачи восстановления
\item Найти общий способ восстановления изображений с вырожденным смазом
\end{list}
\end{block}
\end{frame}



\section{Вспомогательные утверждения}
\begin{frame}
\begin{block}{\emph{Матрица горизонтального циклического смаза} (далее - \emph{матрица смаза}) на $k$ пикселей}
    $$
    C(n, k) = \frac{1}{k}\begin{pmatrix}
          \overbrace{1 \hspace*{2mm} \ldots \hspace*{2mm} 1 \hspace*{2mm} 1}^{k} \hspace*{2mm} 0 \hspace*{2mm} \ldots \hspace*{2mm} 0 \\
          0 \hspace*{2mm} 1 \hspace*{2mm} \ldots \hspace*{2mm} 1 \hspace*{2mm} 1 \hspace*{2mm} \ldots \hspace*{2mm} 0 \\
          \ldots \hspace*{2mm} \ddots \hspace*{2mm} \ldots \hspace*{2mm} \ddots \hspace*{2mm} \ldots \\
          1 \hspace*{2mm} \ldots \hspace*{2mm} 0 \hspace*{2mm} 1 \hspace*{2mm} 1 \hspace*{2mm} \ldots \hspace*{2mm} 1 \\
          \ldots \hspace*{2mm} \ddots \hspace*{2mm} \ldots \hspace*{2mm} \ddots \hspace*{2mm} \ldots\\
          \underbrace{1 \hspace*{2mm} \ldots \hspace*{2mm} 1}_{k-1} \hspace*{2mm} 0 \hspace*{2mm} 0 \hspace*{2mm} \ldots \hspace*{2mm} 1
        \end{pmatrix}
    $$
\end{block}
\begin{block}{\emph{Горизонтальный циклический смаз} (далее - \emph{смаз})}
Умножение изображения на соответствующую матрицу смаза справа есть смаз.
\end{block}
\begin{block}{Утверждение.}
 Матрица смаза всегда представима в виде $FDF^{-1}$, где $D$ - диагональная.
\end{block}

\end{frame}


\section[Метод разбиения]{Метод предобработки в случае большого числа смаза}
\begin{frame}
\begin{block}{Подход}
Главная идея метода - разбить задачу вырожденного смаза на несколько подзадач с невырожденным смазом.
Пусть $n$ - ширина изображения в пикселах, $k$ - число смаза и $d =$ НОД$(n, k)$. Тогда, если $d > 1$, разобьем изображение на $d$ частей, взяв из $d$ подряд идущих колонок ровно одну в каждую часть.

\end{block}

\end{frame}


\section[Метод минимизации]{Метод минимизации погрешности}
\begin{frame}
\subsection{Описание проблемы}
\begin{block}{}
Матрица $C$ в общем случае может быть необратима. Тогда для восстановления необходимо построить приближение $\tilde{C}$ такое, что $C\tilde{C} \approx E$.
\end{block}

\begin{block}{}
Особенности задачи заключаются в наличии большой погрешности (округления и вычислений), возникающей при оцифровке изображений, а также вследствие малых по модулю собственных значений матрицы смаза (плохой обусловленности).
\end{block}

\begin{block}{}
Учитывая все вышесказанное, предлагается строить матрицу $\tilde{C}$, используя ее разложение в диагональную и проебразуя собственные числа, лежащие на диагонали получившейся матрицы.
\end{block}

\end{frame}


\end{document}